\documentclass[a4paper, 11pt, oneside, english]{article}
\usepackage[utf8]{inputenc}
\usepackage[T1]{fontenc}
\usepackage{ebgaramond}

% Load encoding definitions (after font package)

\usepackage{textalpha}

\usepackage{listings}
\lstset{basicstyle=\ttfamily}

% Babel package:
\usepackage[german]{babel}

% With XeTeX$\$LuaTeX, load fontspec after babel to use Unicode
% fonts for Latin script and LGR for Greek:
\ifdefined\luatexversion \usepackage{fontspec}\fi
\ifdefined\XeTeXrevision \usepackage{fontspec}\fi

% ``Lipsiakos"' italic font `cbleipzig`:
\newcommand*{\lishape}{\fontencoding{LGR}\fontfamily{cmr}%
		       \fontshape{li}\selectfont}
\DeclareTextFontCommand{\textli}{\lishape}

\usepackage{booktabs}
\setlength{\emergencystretch}{15pt}
\usepackage{fancyhdr}
\usepackage{microtype}
\begin{document}
\begin{titlepage} % Suppresses headers and footers on the title page
	\centering % Centre everything on the title page
	\scshape % Use small caps for all text on the title page

	%------------------------------------------------
	%	Title
	%------------------------------------------------
	
	\rule{\textwidth}{1.6pt}\vspace*{-\baselineskip}\vspace*{2pt} % Thick horizontal rule
	\rule{\textwidth}{0.4pt} % Thin horizontal rule
	
	\vspace{0.75\baselineskip} % Whitespace above the title

        {\LARGE Zur Darwin'schen Lehre \\} % Title
	
	\vspace{0.75\baselineskip} % Whitespace below the title
	
	\rule{\textwidth}{0.4pt}\vspace*{-\baselineskip}\vspace{3.2pt} % Thin horizontal rule
	\rule{\textwidth}{1.6pt} % Thick horizontal rule
	
	\vspace{1\baselineskip} % Whitespace after the title block
	
	%------------------------------------------------
	%	Subtitle
	%------------------------------------------------
	
	{Von \\\Large Dr. Hermann Eberhard Richter\\} % Subtitle or further description
	
	\vspace*{1\baselineskip} % Whitespace under the subtitle
	
	%------------------------------------------------
	%	Editor(s)
	%------------------------------------------------
	
        %------------------------------------------------
	%	Cover photo
	%------------------------------------------------
		
	%------------------------------------------------
	%	Publisher
	%------------------------------------------------
		
	\vspace*{\fill}% Whitespace under the publisher logo
	
	% Publication year

	{Leipzig, 1865} % Publisher
 
        {\small Jahrbücher der in- und ausländischen gesamten Medizin}

	\vspace{1\baselineskip} % Whitespace under the publisher logo

    Internet Archive Online Edition  % Publication year
	
	{\small Namensnennung Nicht-kommerziell Weitergabe unter gleichen Bedingungen 4.0 International } % Publisher
\end{titlepage}
\clearpage
\setlength{\parskip}{1mm plus1mm minus1mm}
\paragraph{}
Wir haben die Darwin'sche \emph{Streitfrage} bisher in unsern Jahrbüchern unberührt gelassen. Nicht, weil wir sie für eine unwichtige oder für eine unserem ärztlichen Leserkreis fernstehende gehalten hätten. Im Gegenteil! Kein denkender Naturforscher, also auch kein neuzeitlicher Arzt kann sich dieser Frage ganz entziehen. \emph{Einmal} nicht, weil dieselbe tief eingreift in die Fragen über das Wesen der Schöpfung, ob die Welt zufällig oder unendlich ist, also über die Grundfragen einer jeden Naturforschung. --- \emph{Zweitens} nicht, weil Darwin die Entstehung und Fortbildung der Typen, Formen, Abarten und Arten im Tier- und Pflanzenreiche erörtert und damit die Grundlagen jeder wissenschaftlichen Morphologie und Entwicklungs-Lehre, der Hauptbasis der Physiologie, scharfen und immer erneuerten Forschungen unterwirft und sogar noch bis in die Gebiete der Pathologie, der Hygieine und der patholog. Psychologie neue Streiflichter wirft.

Jedoch die D.'sche Lehre ist andrerseits schon seit ein paar Jahren durch ihre Gegner u. Freunde in nichtärztlichen (bez. naturwissenschaftl., theolog., belletristischen) Zeitschriften und Büchern so reichlich besprochen worden, dass wir uns \emph{hier} darauf beschränken können:
\begin{enumerate}
    \item auf eine Anzahl von \emph{Schriften} aufmerksam zu machen, aus welchen unsere Leser am Besten die ganze Lehre und das neuerdings hinzugekommene Material (welches bes. in anthropologischer Hinsicht äußerst interessant ist) kennen lernen werden, --- und

    \item ein Paar \emph{eigene} selbstständige \emph{Bemerkungen} hinzuzufügen, welche diese Streitfrage aufzuklären, vielleicht sogar für manchen gleichgestimmten Denker zum Abschluss zu führen geeignet sind.
\end{enumerate}

\centerline{*\hspace{15mm}*\hspace{15mm}*\hspace{15mm}*\hspace{15mm}*}
\bigskip

\paragraph{}
Die Darwin'sche \emph{Lehre} wird jeder Fachmann am liebsten aus dem Originalwerke selbst (Nr. 1) entnehmen. Die meisten unserer Leser werden aber wahrscheinlich die Rolle'sche \emph{Bearbeitung} (Nr. 2) vorziehen, welche nicht nur gedrängter, sondern auch übersichtlicher und fasslicher geschrieben, zweckdienlicher gedruckt und mit manchen nützlichen Zusätzen versehen ist. Es lässt sich diese Theorie etwa folgendermaassen zusammenfassen (vgl. Nr. 1. S. 528):

Der Schöpfer gab im Anfange einigen wenigen Pflanzen- und Tierformen, vielleicht auch nur \emph{einer} einzigen (3. Aufl.), das Leben, so dass sie wuchsen und sich selbsttätig weiter fortpflanzten bis heutigen Tages. Bei jeder Fortpflanzung \emph{vererbten} sie einerseits ihre wesentlichsten Eigenschaften auf ihre Nachkommen, andrerseits \emph{änderten} diese aber sich in manchen Einzelheiten \emph{ab}, u. zwar in einer für das Individuum entweder schädlichen, oder gleichgültigen, oder nützlichen Weise. Die mit \emph{nützlichen} Einrichtungen versehenen Abarten haben mehr Aussicht fortzubestehen und sich fortzupflanzen. Ihre Abkommenschaft wird im Laufe der Jahrhunderte und Jahrtausende immer zahlreicher werden. (Etwas wie der Viehzüchter und Gärtner die von ihm selbst bevorzugten Rassen und Abarten auswählt und reichlicher erzieht, \emph{züchtet} (daher D.s Ausdruck "`\emph{natural-selection, natürliche Züchtung}"'). Ähnlich wirkt der fortwährend in der gesamten lebenden Welt stattfindende Wettkampf um die Bedingungen der Existenz und des Fortkommens ("`der \emph{Kampf ums Dasein}."' D.), indem derselbe die schwächeren und minderbegabten Arten nach u. nach zum Untergang bringt, hingegen die bevorzugteren nach und nach zu den überwiegenden macht. --- So befindet sich also nach Darwin die gesamte organische Welt von Uranfang an in einer fortgesetzten \emph{Weiterentwicklung} und \emph{Umgestaltung} (durch Vererbung, durch abändernde Einflüsse und durch den Kampf ums Dasein), wobei sich nach und nach, wenn auch erst binnen Hunderttausenden von Jahren, aus anfänglichen Spielarten weiterhin Abarten, Rassen, konstantere \emph{Arten}, ja endlich sogar \emph{Gattungen} und \emph{Familien} bilden, welche immer greller voneinander unterscheidbar werden, jemehr die ehemaligen \emph{Übergangs- und Zwischenformen} zu Grunde gegangen sind.

Dieser Darwin'schen Theorie gegenüber steht die \emph{strenge Artentheorie}, welche in jeder Art des Tier- oder Pflanzengeschlechts eine unwandelbare Einheit sieht, die so geschaffen wurde, wie sie \emph{ist} (wenigstens in allen wesentlichen Eigenschaften) u. die auch \emph{so} dereinst untergehen wird. Man nennt diese Theorie gewöhnlich die Linné'sche, aber zum Teil mit Unrecht. Allerdings hat Linné in der ersten bis fünften Auflage seiner \emph{Genera plantarum} die starre Artentheorie präcis ausgesprochen: "`Species tot sunt, quot diversas formas ab initio produxit Infinitum Ens; quae deinde formae secundum generationis inditas leges produxere plures, \emph{at sibi semper similes, ut species nunc nobis non sint plures quam quae fuere ab initio}."' --- Aber in seinem \emph{reifsten} Alter, in der 6. Auflage der Genera pl., hat er die oben cursiv gedruckten Worte gestrichen und dafür geschrieben: "`\emph{produxere plures, sibi similes, quam quae fuere ab initio}."' Zugleich fügte er zu den Worten "`\emph{diversas formas}"' hinzu "`\emph{et constantes}."' Also hat Linné schon entschieden der \emph{Fortbildungs-Theorie} hinsichtlich der Pflanzenarten gehuldigt.\footnote{S. meinen Codex botanicus Lineanus. Lips. 1835. Fol. S. 9. Nota 3.} --- Der Erste, welcher mit dem starren Artenbegriff gänzlich brach, war Lamarck (\emph{Philosophie zoologique} 1809 und \emph{Animaux sans vertèbres} 1815). Er lehrte, dass sich die Geschöpfe aus niederen zu höheren Formen nach und nach entwickelt haben; zuallererst entstanden Urpflanzen und Urtiere mittels der Generatio aequivoca. Ihm folgten Geoffroy St. Hilaire, Oken u. A. Aber es trat ihnen in Cuvier ein gewichtiger Kämpfer entgegen, welcher durch seine Autorität fast mehr als durch seine Gründe die alte rechtgläubige Lehre von der \emph{Abgeschlossenheit der Species} wiederherstellte, wobei er annahm, dass die \emph{vorweltlichen} durch gewaltsame Erdrevolutionen untergegangen und dass dann durch einen darauffolgenden bewussten Schöpfungsakt \emph{neue} Gattungen u. Arten entstanden seien. --- Nach ihm wagte es zuerst wieder Darwin, die Unwandelbarkeit der Arten und Gattungen zu bezweifeln und so zu sagen die ganze organische Schöpfung in Fluss zu bringen. Diess war ihm jedoch nur erst dadurch möglich, dass vorher Lyell die ganze Lehre von den plötzlichen gewaltsamen Erdumwälzungen gestürzt und mit immer zahlreichern, immer siegreicheren Gründes bewiesen hatte, dass alle Hauptveränderungen unseres Erdballes ganz allmählig im Laufe der Jahrtausende und Jahrmillionen durch \emph{Ursachen} hervorgebracht worden sind, \emph{welche den jetzt wirkenden ganz oder fast ganz gleichkommen}. --- Diese Lyell'sche geologische Theorie war aber wiederum nur erst begründbar, nachdem die \emph{Astronomie} bewiesen hatte, dass die Welt in Raum und Zeit \emph{unendlich} sei, dass Millionen Jahre in ihr gar Nichts, in der Geschichte unseres Sonnensystems nur wenig zu bedeuten haben, und dass die Weltkörper sich in einem fortlaufenden Aus- und Umbildungsprozess befinden, welcher für unser Sonnensystem und verwandte darin besteht, dass ein feiner Nebel aus Weltstaub sich allmählig nach Attraktionsgesetzen zu festen Massen verdichtet (Laplace).

Vergleicht man obige beide Theorien der Artenentstehung, --- welche im Einzelnen noch der mannigfachsten Abänderungen und Zusätze fähig sind, --- hinsichtlich ihres \emph{Wertes für die Wissenschaft} miteinander: so sieht man bald ein, dass die \emph{Fortentwicklungs-Lehre} die entschiedensten Vorzüge vor der Theorie der \emph{ein für allemal geschaffenen Arten} besitzt. Denn die letztere lähmt den menschlichen Forschungsgeist. Man wird erdrückt von der unendlichen Menge der Formen und Lebensweisen im Tier- und Pflanzenreiche. Man fühlt sich unfähig, die vernünftigen Gründe (den sogen. \emph{Schöpfungsplan}) zu finden, nach denen diese Millionen willkürlich geschaffen und dann wieder nach Ausweis der Fossilien vernichtet worden sind. Man wird also früher oder später auf demjenigen Standpunkt ankommen, welchen die älteste menschliche Kulturstufe, die \emph{ostindische} einnimmt, indem sie es für die höchste Weisheit hält, in Bewunderung der unendlichen Schöpfung die Hände über dem Bauche gefaltet da zu sitzen und "`\emph{om, om}"' zu rufen.

Hingegen die Theorie, \emph{dass die Arten, Gattungen, Familien u. s. w. sich selbsttätig eine aus der andern im Laufe der Zeit herausentwickelt haben und fernerhin sofort umbilden werden}: diese weckt vom ersten Augenblicke an Differenz in der Wissenschaft und ist geeignet, so lange es überhaupt Forscher gibt, zu den eingehendsten Untersuchungen im Bereiche der Organismen anzuregen und sogar zu zwingen. Denn hier gibt es \emph{Natur-Gesetze} zu erlauschen, welche, eines mit dem andern innig verbunden, aus den verschiedensten Gebieten der Wissenschaft herbeigeleitet werden müssen. --- Gesetze, welche bald durch systematische Studien lebender oder vorweltlicher Pflanzen- und Tierarten, bald durch anatomische Zergliederungen, bald durch physiologische Beobachtungen u. Experimente, bald durch landwirtschaftliche (z. B. Züchtungs- und Acclimatisations-) Versuche u. s. w., kurz durch eine unendliche Reihe der eingehendsten naturwissenschaftlichen Arbeiten gefunden werden können, aus deren Betriebe noch nebenbei ein reicher Schatz verschiedener, nicht zu dieser Frage gehöriger Entdeckungen gewonnen werden würde.

Wer nur irgendeinmal, in iirgendeinemKapitel der \emph{Botanik} oder \emph{Zoologie}, spezielle Formenunterscheidung, d. h. \emph{Artenbestimmung} betrieben hat, der wird das hier Gesagte lebhaft mitfühlen u. erfassen. Wenn ich mich Beispielsweise aus meiner botanischen Jugendzeit der Gattungen Rosa, Rubus, Viola, Salix u. s. w. erinnere, --- wie da, je mehr man forscht und unterscheidet, je mehr Sorten man ins Herbar einsammelt, desto mehr die Typen auseinander gehen oder ineinander fließen, wie da ein leitendes Princip, eine Einhaltung geschlossener Arten und wohlgeordneter Unterarten nach und nach immer schwieriger zu werden scheint, wie endlich manchmal kein einzelner Forscher mehr mit dem Andern einig werden kann:--- in der Tat, da muss die Darwin'sche Theorie wie ein Lichtblitz, wie eine geistige Befreiung erscheinen.

Ein sehr hübsches Beispiel dieser Art gibt das Schriftchen von Fritz Müller (Nr. 3). Dieser gediegene Naturforscher beschäftigte sich eben mit dem Studium der Formen u. Lebensweisen gewisser Krustentiere, als ihm die Darwin'sche Lehre über eine Menge bis dahin unerklärlicher Tatsachen frappanten Aufschluss gab. Ein Paar von diesen sind so nett, dass wir sie hier kurz mitteilen. Bei einer gewissen Art von Scheeren-Assel zeigen sich die reifen Männchen in zwei verschiedenen Gestalten, Abarten, wenn man will. Die eine Partei hat gewaltige, langfingerige und leichtbewegliche Scheeren, die andere besitzt nur kleine plumpe Scheeren, aber dafür sehr vollkommen gestaltete und mit weit zahlreichern Riechfäden versehene Fühler. Beide sind offenbar nach Darwin bevorzugte Abänderungen: Erstere begabter ihre Beute festzuhalten und größere Tiere zu bewältigen: Letztere geeigneter zum Aufspüren ihrer Nahrung. (Also analog den zwei Formen des Wolfs, von denen Darwin berichtet: die eine langbeinig, windhundartig, zum Verfolgen des Wildes geschickter, die andere plump und kräftiger, in Schafherden einbrechend.) --- Bei mehreren Krabben findet sich nur \emph{eine} Kneifzange, während der entsprechende Vorderfuß der andern Seite zu einer Art von Lauf- oder Ruderfuss verkümmert ist: eine Einrichtung, welche dem Nahrungsbetrieb dieser Geschöpfe so günstig gewesen zu sein scheint, dass sie nach und nach allgemeine Regel geworden ist. --- Bei andern Arten finden sich auf dem Rücken mancher Weibchen eigentümliche Hervorragungen, Haftorgane, an denen sich das Männchen bei der Begattung festhalten kann; diess ist natürlich der Fruchtbarkeit sehr dienlich, und so werden mit der Zeit die solcher Haftorgane entbehrenden Weibchen immer seltener werden. --- Noch beweiskräftiger sind vielleicht die von Fr. M. beigebrachten Tatsachen aus der \emph{Entwicklungsgeschichte} dieser Tiere, deren Jugendzustände durchaus heterogene, scheinbar ganz andern Gattungen und Familien angehörige Formen darstellen. Es ist dabei hervorzuheben, dass M. mehre dieser Entdeckungen erst durch die Anleitung der Darwin'schen Lehre aufgefunden hat. Doch muss diess unser Leser in dem kleinen, aber gehaltreichen Buche selbst nachlesen, welches seinen Zweck "`diese Frage durch Herbeischaffen neuen verwertbaren Stoffes allmählig spruchreifer zu machen,"' in sehr befriedigender Weise erfüllt; welches auch ein paar das D.'sche Gesetz weiter ins Einzelne ausbauende \emph{Sätze} aufstellt.

In der Tat darf man nur die großen \emph{Formenwechsel} betrachten, welche die \emph{Entwicklungsgeschichte} jedes einzelnen Geschöpfes, vom Pflänzchen bis zum Menschen, und insonderheit der \emph{Generationswechsel} in den sogen. niederen Klassen des Tier- und Pflanzenreichs durchläuft, um sich zu überzeugen, dass die Natur darin Außerordentlicheres leistet, unbegreiflichere Sprünge macht, als es die Darwin'sche Lehre irgend unserer Phantasie zumutet. Welch ein ungeheurer Unterschied ist nicht z. B. zwischen der frei umherschwimmenden Meduse und dem festsitzenden Keulenpolypen, aus dem jene sich entwickelt hat, und welcher seinerseits wieder aus einem freischwimmenden gewimperten Infusorium entstand. Welcher Unterschied zwischen dem Bandwurme und seiner Finne, zwischen der freilebenden Cercaria und den eingeschlossenen (oder ganz eingebalgten) Distomen, zwischen dem Schimmelfaden-Mycelium u. den daraus erwachsenden Champignons, zwischen den Thallus ähnlichen Anfängen der Farrenkräuter und deren späterer, bis zur Palmenform gedeihender zweiter Brut, zwischen den fertigen Algen und deren verschiedenen Ur- und Teilungsformen u. s. w.

Fast noch mehr als die neuzeitliche, musste das Studium der \emph{vorweltlichen Tier- und Pflanzenwelt} durch Darwins Lehre einen gewaltigen Anstoß erhalten, welcher viele bisher unlösbar erschienene Fragen einer wissenschaftlichen Aufklärung zuzuführen verspricht. Darwin hat daher auch zunächst in den Kreisen der \emph{Geologen} den lebhaftesten Kampf angeregt und bedeutende Anhänger gewonnen. So hat D.s Bearbeiter Rolle (Nr. 2) aus seinen eigenen und seines Freundes, Gust. Jägers Arbeiten mehre bedeutende Tatsachen beigetragen. --- Am entscheidendsten jedoch ist der Beitritt des großen Geologen, Ch. Lyell, dessen oben angezeigtes Werk (Nr. 4) ein Muster von naturwissenschaftlicher Bearbeitung, eine Fundgrube der interessantesten Tatsachen und zugleich eine weitergeführte, tieferbegründete Entwicklung der Darwin'schen Ideen ist. Die der letzteren gewidmeten \emph{Schlusskapitel} (Nr. 4. S. 318 u. 453) gehören zu dem Besten, Was über diese und die verwandten Theorien geschrieben worden ist. Lyell unterscheidet als zwei der Lamarck'schen Lehre entsprungene Zweige: die \emph{Fortschrittstheorie} (von Sedgwick, Miller, Owen, Bronn, Brogniart u. s. w.), welche annimmt, dass nach einem vorausbestimmten Vervollkommnungsplan sich eine Gattung nach der andern, eine Familie oder Klasse nach der andern gebildet habe, daher die Wirbeltiere nach den Wirbellosen, zuletzt die Säugetiere und der Mensch, --- und die \emph{Umwandlungstheorie} von Darwin und Wallace, welche die Abänderungen der Artentypen durch natürliche Auswahl u. Kampf ums Dasein erklärt, ohne notwendigerweise ein Fortschrittsgesetz herbeiziehen zu müssen. L. wägt aufs Gewissenhafteste die für oder gegen beide Lehren sprechenden Tatsachen ab und entscheidet sich für die Darwin'sche. Zu deren Veranschaulichung wählt L. (Cap. 23. S. 394) ein äußerst glückliches (wenn wir nicht irren, von unserem Landsmann Prof. Max Müller in Oxford entlehntes) Beispiel: nämlich die im Laufe der Jahrhunderte geschehene \emph{Umwandlung einer Sprache in die andere} (z. B. des römischen Lateins ins Italienische, Französische, Spanische, des Deutschen ins Englische u. s. w.), indem er dartut, wie auch hier die Zwischenstufen ausgestorben sind und daher gegenwärtig die einzelnen Sprach-Arten unter sich und von ihren Stammdialekten total verschieden erscheinen. [Z. B. das Sächsische der Siebenbürgen wird von keinem dermaligen Deutschen, das alte Gotisch der Isländer von keinem heutigen Norweger mehr verstanden. R.]

Derselbe Lyell hat auch richtig gefühlt, wie wichtig die Darwin'sche Lehre für die gesamte \emph{Anthropologie}, insbesondere für die Frage \emph{über die Abstammung u. das Alter des Menschengeschlechts} werden müsse. Letzteres ist ja das eigentliche Thema seines Buches, in dessen ersten 2 Drittteilen (S. 10-317) der Vf. die neuesten Funde von menschlichen Knochen und Kunsterzeugnissen in den ältesten diluvialen sog. nachpleiocenen und sogar in den jüngsten tertiären Schichten, nebst den für ein solch hohes, nach Hunderttausenden von Jahren zu bemessendes Alter der Menschen sprechenden anderweiten Tatsachen beibringt und erörtert. Hier sind es besonders die beiden uralten und auffallenden \emph{Schädel aus dem Neandertal} u. \emph{von Engis}, welche an die Lehre Darwins erinnern. Denn der erstgenannte, obschon unzweifelhaft ein Menschenschädel, steht doch in manchen Beziehungen den höheren Affengattungen fast näher als einem heutigen europäischen Schädel. Selbet die niedrigste Schädelform heutiger Völker, die des Australnegers auf Neuholland, steht noch weit über dem Neandertalschädel an Fülle des Vorderhirns. Beide Rassen aber haben oder hatten offenbar das gemeinsame Schicksal, durch höher organisierte, hirnbegabtere Menschenstämme ausgerottet zu werden.\footnote{Vgl. über Rassenschädel: Jahrbb. 125. 337.} Ein gleiches Loos hat aber auch die Nachfolger des Neandertalers betroffen: einen kurzköpfigen, feingliedrigen, zwerghaften, den heutigen Lappländern ähnlichen Menschen-Stamm, welcher einst die Küsten Deutschlands, Skandinaviens und vielleicht auch Frankreichs bewohnt hat.

Durch besagte Schädel ist nun auch ein nüchterner Fach-Anatom, Dr. Huxley, in vorliegende Frage mitverwickelt und, fast möchte man sagen wider Willen, zu einem entschiedenen Anhänger der Darwin'schen Lehre geworden, welcher deren Anwendbarkeit auf spezielle \emph{zootomisch-physiologische} Aufgaben tatsächlich beweist. Seine Schrift beginnt mit einer Naturgeschichte der menschenähnlichen Affen und erörtert dann im zweiten Abschnitte gründlich, aber ganz unparteiisch, dass \emph{Für} und \emph{Wider} der Behauptung, dass der Mensch ebenfalls aus einem affenähnlichen Geschöpf, oder Mensch und Affen aus einem gemeinsamen Grundstamme hervorgegangen seien. Er kommt zu dem Ergebnisse: "`wir mögen Organe vornehmen, welche wir wollen, stets werden wir die anatomischen Verschiedenheiten, welche den Menschen vom Gorilla und Chimpanze scheiden, nicht so groß finden, als die, welche den Gorilla von den niedrigeren Affen trennen."' --- Im 3. Kapitel ist dann den obengenannten Neandertaler und Engis-Schädeln eine ebenfalls sehr genaue Untersuchung gewidmet, welche herausstellt, dass namentlich ersterer der affenähnlichste aller bekannten Schädel sei, nichtsdestoweniger aber entschieden einem Menschen, nicht aber einem Affen oder einem zwischen beiden mitteninne stehenden Geschöpf angehört habe.

Geh.-Rath Mayer widerspricht in dem angef. Aufsätze (Nr. 6) den Ansichten von Lyell, Darwin und Huxley insoweit, als er das Alter des Menschengeschlechts nicht auf 100,000 und mehr Jahre, sondern nur etwa auf 7-8000 zurückführt u. die affenähnliche Struktur der genannten Schädel, daher die Abstammung der Menschen aus pithekoïden Geschöpfen leugnet. Er gibt aber zu, dass der Mensch in der Diluvial-Zeit gemeinsam mit den großen, jetzt ausgestorbenen Dickhäutern und Raubtieren gelebt habe. Er erklärt Darwins Lehre für unbegründet, weil ein Wesen nicht den Keim eines andern von sich verschiedenen in und aus sich erzeugen \emph{könne}, --- und weil diese Lehre der aus dem Universum uns entgegen leuchtenden Allmacht oder Schöpfung von Wesen in unendlicher Mannigfaltigkeit \emph{unwürdig sei}. [Scheint uns keine naturwissenschaftliche Argumentation.]

Neuerdings hat Huxley im Archiv f. Anat. etc. Hft. 1. p. 1. 1865 das Neueste über den Neandertaler Schädel (\emph{Homo neanderthalensis} von King als besondere Species unterschieden!) mitgeteilt u. dabei die wenig wissenschaftlichen Gegengründe Mayers kräfig abgefertigt.

Karl Vogts \emph{Vorlesungen} (Nr. 7.) bewegen sich ebenfalls in dem Kreise der durch obengenannte Schädelbefunde und durch die Darwin'sche Hypothese angeregten Streitfragen. Er bereichert und vervollständigt dieselben mit einer Menge neuer, durch die ausgezeichnetsten Naturforscher unserer Zeit und durch eigene glückliche Forschungen gewonnener Tatsachen. Er behandelt im \emph{ersten} Bande die Anatomie des Menschen, bes. soweit sie für Rassenunterscheidung Wert hat, die verschiedenen Messungsmethoden und deren Ergebnisse, die Unterschiede und Ähnlichkeiten der Affen und Menschen und die daraus hervorgehenden Folgerungen. Im \emph{zweiten} Teile sind die Tatsachen über die Urzeit des Menschengeschlechts, über die ältesten Befunde von Menschenknochen und Menschenwerken (Stein- und andern Geräten, Küchenabfällen, Gräbern, Pfahlbauten, Haustieren u. s. w.) zusammengestellt; worauf Vf. wieder auf die Gesetze der Rassenbildung zurückkommt u. die wichtigsten Sätze über \emph{die Rassenbildung im Menschengeschlechte} in 6 \emph{Punkten} formuliert (2. S. 248). Er schließt endlich mit einer bündigen Darlegung und Verteidigung der Darwin'schen Lehre. Nachdem er schon vorher (2. S. 254) dem Schwann'schen Satz: "`\emph{Jeder pflanzliche und tierische Organismus entwickelt sich aus einer einzigen Zelle},"' gehuldigt hat, erklärt V. sich weiterhin mit D. einverstanden, dass auch die zusammengesetzteren Tiere und Pflanzen aus \emph{einfachen Urzellen} herausgebildet worden seien; er nimmt aber an, dass diese Zellen \emph{von Anfang an unter einander verschiedene} gewesen seien hinsichtl. ihrer Zusammensetzung, ihrer Lebensweise, ihrer Fortpflanzung, ihrer äußern Gestalt: dass also \emph{nicht} eine einzige Zellenform als Grundtypus und Uranfang der gesamten organischen Schöpfung anzusehen sei (2. S. 276). --- V. nimmt (2. S. 272) an, dass die Erzeugung neuer Mischrassen besonders in solche Zeiträume falle, wo sich die äußern Verhältnisse, die umgebenden Mittel abänderten u. dadurch die bisherige, unter stetigeren äußern Umständen festgewordene Starrheit des Typus brachen. --- Wie z. B. in dem von Lovén berichteten Falle, wo einige, aus der ehemaligen Salzflut zurückgebliebene Seekrebse sich in dem süßen Wasser des Wener- und Wettersees bis heutzutage erhalten, jedoch ihre Form bedeutend abgeändert haben.\footnote{Ein soeben in einem Heft ausgegebenes neueres Werk von Dr. Fr. Rolle "`\emph{der Mensch, seine Abstammung und Gesittung im Lichte der Darwin'schen Lehre von der Art-Entstehung und auf Grundlage der neuern geologischen Entdeckungen dargestellt} (Frankf. a. M. 1865. J. Chr. Hermann'sche Verlagsbuchh. 8.), kündigt sich als eine selbstständige Fortsetzung des früheren Werkes (Nr. 2) an u. will die Anwendung der D.'schen Lehre auf den Menschen, seine Abkunft, die Grundlagen seiner körperlichen und geistigen Charaktere und die Entwicklung seiner Gesittung populär-wissenschaftlich entwickeln, --- also eine Anthropologie nach D.'schen Grundsätzen, d. h. nach der Regel, "`\emph{dass man beim Menschen so gut wie bei der übrigen Lebewelt unserer Erde natürliche Dinge auch nach natürlichen Gesetzen zu erklären hat}."' (Verf.)}

Wie im Vorstehenden die Darwin'schen \emph{Grundsätze} für die allgemeinere \emph{Anthropologie} verwertet wurden, so werden sie bald nicht minder belangreich für die eigentliche \emph{Heilkunde} werden. Schon für \emph{Anatomie} und \emph{Physiologie}, für die Lehre von der Entwicklung der Formen im einzelnen Individuum wie in der Gesamtspecies, für die anatomischen Abweichungen bis zu den Missbildungen, für die Variationen einzelner physiologischer Vorgänge unter verschiedenen äußern oder Inneren Verhältnissen, wird das durch D. belebte Studium der Artenabänderung wichtig werden, sobald die Naturforscher dasselbe auf das ganze Tier- u. Pflanzenreich, die Landwirte auf alle Kulturpflanzen und Nutztiere ausgedehnt haben werden. --- Von da aus dürfte es bald in die \emph{Hygieine} u. \emph{Naturtherapie} übergreifen (Züchtung, Fütterungsweisen, Wohnungen, Akklimatisierung u. s. w.). --- Nächstdem wird die gesamte \emph{Pathologie} wesentlich von den D.'schen Prinzipien beeinflusst werden: umso mehr, da dieselbe sich jetzt eben der \emph{ätiologischen Pathogenie}, der Erforschung der krankmachenden \emph{Ursachen}, zuwendet (s. Jahrbb. 111. 115.; 116. 351. flg.). Sind ja doch alle Erkrankungen mehr oder weniger ein Ergebnis des "`\emph{Kampfes ums Dasein}"' und vor Allem die en- und epidemischen Krankheiten von diesem Standpunkt aus zu erfassen! Hier werden einige der von D. beigebrachten Belege unmittelbar passen, andere zu treffenden Analogien führen. --- Der Satz von den "`\emph{durch vererbte Eigenschaften bevorzugten oder benachteiligten Rassen}"' wird hier ausgedehnte Anwendung finden. Wir werden danach z. B. begreifen, dass eine mit sackförmigen Blinddärmen ausgestattete Familie nach und nach ausstirbt (durch häufige Entstehung von Darmverschlingungen), oder dass eine mit langen Phimosen begabte Rasse an Fruchtbarkeit hinter den mit freier Eichel Versehenen oder Beschnittenen zurückbleibt u. dgl. m. --- Noch auffälliger wird sich Diess vielleicht für \emph{Phrenologie} und \emph{Psychopathologie} bewähren. Es deuten ja schon die obenerwähnten Autoren (Lyell u. s. w.) an, dass die urältesten Rassen von Neandertal u. Engis in Folge ihrer nachteiligen Hirnorganisation untergegangen sein mögen, indem sie begabteren Volkstämmen unterlagen. (Ähnlich in historischer Zeit die Quancho's, die Westindier, jetzt die Australneger.) Es lehrt uns schon die Erfahrung \emph{eines} Menschenalters, wie die Familien des Säufers, des Spielers, des Lüderlichen zu Grunde gehen. Die Geschichte erzählt uns, durch welche geistigen Eigenschaften die Herrscherfamilien der Julier (Caesar, August, Tiber u. flg.), der Stuarts, der Bourbonen u. s. w. untergegangen sind. Ähnliche Bemerkungen kann man sehr leicht an andern namhaften Geschlechtern machen. Wenn wir nach tausend oder zehntausend Jahren wiederkehren könnten, so würden wir vielleicht die Genugtuung haben, zu finden, dass die schlechten Charaktere aller Art, die Gewalttätigen und Unterdrücker, die Verdummungsmänner, die Schmarotzer u. s. w. durch ihre Charakterfehler selbst zu Grunde gegangen seien und die ehrlichen Leute die Majorität erlangt haben! --- Inzwischen wird der \emph{Zukunfts-Medicin} wenigstens durch den \emph{Darwinismus} eine höhere Aufgabe als bisher zugewiesen: nämlich dem Menschengeschlechte seinen Kampf ums Dasein zu erleichtern und hinzuwirken auf Veredlung seiner körperlichen und geistigen Begabungen durch rationelle Züchtung.

Bei dieser mehrseitigen Bedeutung des Darwin'schen Ideenganges für denkende Ärzte ist es erfreulich, dass schon ein \emph{Mediciner}, und zwar einer der berühmtesten, demselben seine Aufmerksamkeit gewidmet hat, wenn gleich in dissentierender Weise. Es ist diess Prof. Kölliker (Nr. 7.). Derselbe beginnt seinen Vortrag mit Aufzählung der acht gewichtigsten \emph{Einwände}, welche man gegen Darwin vorgebracht hat und nennt es mit Recht einen Fehler dieser Theorie, dass sie im Wesentlichen eine \emph{teleologische} sei. Da er, K., sich nicht denken kann, dass die Organismen, höhere wie niedere, als sogleich vollendete Formen "`\emph{en bloc erschaffen seien},"' oder dass die höheren durch \emph{Generatio aequivoca} sich unmittelbar aus einer organisationsfähigen organischen Materie hätten (fix und fertig) bilden können: so bleibt ihm nur "`\emph{die Schöpfungstheorie durch Generatio secundaria}"' als möglich denkbar: "`dass nur Eine oder wenige Grundformen selbstständig und unabhängig entstanden seien, aus denen alle übrigen durch weitere Entwicklung hervorgingen."' Diese Generatio secundaria könnte nun geschehen sein: "`1. entweder nach Darwins Princip der natürlichen \emph{Züchtung} allmählig, oder 2. durch langsamere oder sprungweise Veränderungen unter Einwirkung eines die ganze Natur beherrschenden \emph{Entwicklungsgesetzes}."' Letzteres nennt K. "`\emph{die Theorie der heterogenen Zeugung}."' Nach dieser hätten die Geschöpfe die Fähigkeit, aus von ihnen erzeugten Keimen andere \emph{abweichende} Geschöpfe hervorzubringen, und zwar in zweierlei Weise: \emph{entweder} (nach Analogie des \emph{Generationswechsels}) indem die befruchteten Eier bei ihrer Entwicklung in [andere, von den Eltern verschiedene] höhere Formen übergingen, \emph{oder} (nach Analogie der \emph{Parthenogenesis}), indem die primären und späteren Organismen ohne Befruchtung andere Organismen erzeugten. Damit statuiert nun K. auch, anstatt Darwins allmähliger Umbildungen, viele \emph{sprungweise} Veränderungen. Er verwirft die natürliche Züchtung und die Bevorzugung nützlicher Abarten und setzt an deren Stelle "`\emph{einen zu Grund liegenden Entwicklungsplan}."' Gerade hierin aber scheint mir kein Fortschritt, sondern ein Rückschritt zu liegen; so ein geheimnisvoll entworfener Plan muss in untergeordneten Köpfen unfehlbar wieder zu mystischen Ansichten in der Naturwissenschaft führen. Darwin dagegen stellt uns ein paar greifbare, der materiellen Forschung zugängliche Ursachen des Formenwechsels hin, denen wir nach Belieben oder Glück noch mehre andere natürliche Ursachen hinzuzufügen unbehindert sind. Er spornt uns zum Forschen an. K.s Entwicklungsplan würde bald seine lähmende Einwirkung merken lassen!

Im Gegenteil erscheint \emph{uns} sogar die Darwin'sche Lehre noch in \emph{dem} Punkte \emph{schwach} zu sein, dass sie \emph{nicht radikal genug ist}. Bekanntlich ist diess ziemlich allen englischen Forschern und Denkern, seit und mit Baco von Verulam, eigentümlich. "`Die Engländer philosophieren nur bis auf einen gewissen Punkt, bei welchem sie stehen bleiben,"' schreibt schon Moses Mendelssohn (Briefe an Lessing, den 27. Febr. 1758). Aber auch unter den zahlreichen Anhängern D.s ist, soweit uns bekannt, keiner entschieden folgerichtig auf den Grund des ganzen Gebäudes hinabgegangen.

Wenn D. die Arten, Gattungen, Familien und Klassen eine aus der andern durch Umwandlung in Folge \emph{natürlicher Ursachen}, nämlich innerer Selbsttätigkeit und äußerer Einwirkungen, hervorgehen lässt: so schlägt er weiterhin seine eigene Theorie wieder tot, wenn er zu Anfang dieser Reihe 5, 4 oder noch weniger \emph{ursprünglich erschaffene} Urformen annimmt. Denn diejenige Schöpferkraft, welche eine, zwei oder mehr solcher \emph{Urzellen} (um mit Vogt zu reden) erschuf, kann auch Millionen verschiedener Arten und sehr zusammengesetzte Organismen aus dem Nichts hervorgerufen haben. Das ist dann ganz egal! Nicht weiter kommen wir mit der Annahme einer vorher existiert habenden organischen Urmaterie (des Oken'schen \emph{Urschleims}), aus welcher sich dann die ersten einfachsten Organismen mittels der \emph{Generatio aequivoca} gebildet haben sollen:--- eine Ansicht, welche schon seit längerer Zeit von K. H. Baumgärtner Freiburg\footnote{Physiologische Briefe. In dessen Vermächtnissen eines Klinikers. Freiburg 1862. 8.} verfochten wird. --- Immer bleibt hier die \emph{Hauptsache} unbegreiflich: \emph{wie entstand die erste Zelle aus dem bisher Unorganischen?} Immer bleibt hier Nichts übrig, als eine willkürliche planmäßige Erschaffung, auf welche wir daher auch die Deutschen Vogt und Kölliker zurückkommen sehen. --- Es handelt sich nun darum, ob die Wissenschaft nicht auch diesen außerordentlichen Eingriff in dem allg. Entwicklungsgang entbehren kann, d. h. ob in der \emph{Natur} nicht allezeit Alles \emph{natürlich} zugegangen ist?

Um recht folgerichtig in dieser Streitfrage zu urteilen, muss man auf deren \emph{Urgrund} zurückgehen, welcher (wie wir Eingangsweise zeigten) in der durch die neuere Astronomie begründeten Weltanschauung liegt.\footnote{Vergl. hierzu das interessante und tatsachenreiche, dabei streng religiös gehaltene Werk von Camille Flamarion "`\emph{die Mehrheit bewohnter Welten.}"' Deutsch von Dr. Adolph Drechsler. Leipzig 1865. J. J. Weber. 8.} Dieser zufolge ist die Welt \emph{unendlich} in Zeit und Raum. Sie hat niemals angefangen, sondern seit Ewigkeit bestanden. Sie wird nie authören; Stoff und Kraft sind unvergänglich. Das Einzige, was sich unaufhörlich in ihr ändert, ist die \emph{Form}. Es entstehen fortwährend neue Gebilde und vergehen alte. Der Weltraum ist erfüllt mit werdenden, reifen und absterbenden Weltkörpern, wobei wir unter \emph{reifen} diejenigen verstehen, welche fähig sind, lebende Organismen zu beherbergen. --- Demnach halten wir auch das Dasein \emph{organischen Lebens} im Weltreich für ewig; es hat immer bestanden und hat in unaufhörlicher Folge sich selbst fortgepflanzt, und zwar \emph{in organisierter Form}, nicht als ein mysteriöser Urschleim, sondern in Gestalt lebender Organismen, als \emph{Zellen} oder aus Zellen zusammengesetzte \emph{Individuen. Omne vivum ab aeternitate e cellula!}

Damit erledigt sich sogleich die Frage, auf welche Weise die \emph{ersten} Organismen in die Welt gekommen seien? Da es deren immerdar irgendwo in der Welt gegeben hat, so fragt es sich blos: "`wie sind sie zuerst \emph{auf diesen oder jenen Weltkörper}, nachdem er bewohnbar geworden, \emph{hingelangt?}"' Und da antworten wir kühn: "`\emph{aus dem Weltraume!}"'

Die Astronomie zeigt, dass im Weltraume Unmassen feiner Substanzen schweben: von den fast körperlosen Kometenschweifen bis zu den in unserer Atmosphäre erglühenden und häufig auf die Erde fallenden Meteorsteinen. In letzteren hat die Chemie außer den geschmolzenen Metallen noch Reste von \emph{organischer Substanz} (Kohle) nachgewiesen Die Frage, "`ob diese organischen Stoffe, bevor sie durch Erglühen des Aëroliths zerstört wurden, aus formlosem Urschleim oder aus geformten organischen Gebilden bestanden haben?"' ist jedenfalls für Letztere zu entscheiden. Denn dafür haben wir eine entsprechende Erfahrung in unserer eigenen Atmosphäre. Überall, wo wir hinreichende Luftmengen durch Baumwolle filtrieren, da finden wir mikroskopische organ. Körper, besonders Pilzkeime und Infusorien in derselben. Nach Ehrenbergs Entdeckungen führt der Aequatorialwindstrom unendliche Mengen sog. Infusorien-Staubes aus Afrika u. Südamerika hoch über die Alpen und Pyrenäen hinweg nach Mitteleuropa herunter. Die als \emph{roter Schnee} bekannten Infusorien (s. Agassiz geolog. Alpenreisen, herausg. von C. Vogt. Frankf. a. M. 1844. S. 235 fig. Tab. 1. 2), welche sich auf den Schneefeldern der Hochalpen in weiter Ausdehnung oft binnen wenig Tagen bilden, haben vielleicht denselben Ursprung. Denn sie besitzen schon eine allzukomplizierte Organisation, um aus blosem \emph{Urschleim} entstanden zu sein; auch ist nicht zu begreifen, wie ein solcher auf die Alpengipfel hinaufgelangt oder dort erzeugt sein soll. --- Wenn nun aber einmal mikroskopische Geschöpfe so hoch in der Atmosphäre der Erde schweben: so können sie auch gelegentlich, z. B. etwa unter Attraktion vorüberfliegender Kometen oder Aërolithen, \emph{in den Weltraum gelangen} und \emph{dann auf einem bewohnbar gewordenen}, d. h. der gehörigen Wärme u. Feuchtigkeit genießenden, \emph{andern Weltkörper aufgefangen, sich durch selbsteigene Tätigkeit weiterentwickeln}.

Diese Hypothese ist klar und einfach; sie lässt sich naturwissenschaftlich erörtern und ausbilden; sie steht im Einklang mit den auf andern Gebieten der Naturwissenschaft eingebürgerten Anschauungen; sie liefert den Schlussstein zu Darwins kühnem Gebäude.
\clearpage
\begin{enumerate}
    \item Charles Darwin, \emph{Über die Entstehung der Arten im Tier- u. Pflanzenreiche durch natürliche Züchtung, oder Erhaltung der vervollkommneten Rasse im Kampfe ums Dasein}. Nach der 3. engl. Ausgabe aus dem Englischen übersetzt und mit Anmerkungen versehen von H. G. Bronn. 2 \emph{verbesserte und sehr vermehrte Auflage}. Mit Darwin's Portrait in Photographie. Stuttgart 1864. Schweitzerbart. 8. 8 n. 551 S.

    \item Ch. Darwins \emph{Lehre von der Entstehung der Arten im Pflanzen- und Tierreich in ihrer Anwendung auf die Schöpfungsgeschichte} dargestellt und erläutert von Dr. Friedr. Rolle. Mit Holzschnitten. Fraukfurt am Main 1863. Jo. Chr. Herrmann'sche Verl.-Buchh. 8. 7 u. 274 S.

    \item \emph{Für Darwin}. Von Fritz Müller. Mit 67 Figuren in Holzschnitt. Leipzig 1864. Wilh. Engelmann. 8. 2. u. 91 S. [Mit dem Titelmotto: "`Caeterum, nullius in verba jurans, aliorum inventa consarcinare haud institul; quae ipse quaesivi, reperi, repetitis vicibus diversisque temporibus observavi, propono."' O. F. Müller, histor. vermium.]

    \item Sir Charles Lyell, \emph{Das Alter des Menschengeschlechts auf der Erde und der Ursprung der Arten durch Abänderung, nebst einer Beschreibung der Eiszeit in Europa und Amerika}. Nach dem Englischen mit eigenen Bemerkungen u. Zusätzen u. in allgemein verständlicher Darstellung von Dr. Louis Büchner (etc.). Autorisierte deutsche Übertragung nach der \emph{dritten Auflage} des Originals. Mit zahlreichen Holzschnitten. Leipzig 1864. Theodor Thomas. 8. 9 n. 472 S.

    \item Thomas Henry Huxley, \emph{Zeugnisse für die Stellung des Menschen in der Natur. Drei Abhandlungen: Über die Naturgeschichte der menschenähnlichen Affen. Über die Beziehungen des Menschen zu den nächstniedrigen Tieren. Über einige fossile menschliche Überreste}. Aus dem Englischen übersetzt von J. Victor Carus. Mit in den Text eingedruckten Holzschnitten. Braunschweig 1863. Fr. Vieweg u. Sohn. 8. 6 und 173 S.

    \item \emph{Zur Frage über das Alter und die Abstammung des Menschengeschlechts}. Von Geh. Med.-Rat und Prof. Mayer in Bonn. In Reichert und Du Bois-Reymond Archiv für Anat., Physiol. u. wiss. Med. Jahrg. 1864. Hft. 6. p. 696-728.

    \item Karl Vogt, \emph{Vorlesungen über den Menschen, seine Stellung in der Schöpfung und in der Geschichte der Erde}. Gießen 1863. J. Ricker'sche Buchhandl. 1. Bd. 15 u. 298 S., --- 2. Bd. 8 u. 328 S.

    \item A. Kölliker, \emph{über die Darwin'sche Schöpfungs-Theorie}. Eis am 13. Februar 1864 in der phys. med. Gesellschaft von Würzburg gehaltener Vortrag. Leipzig 1864. W. Engelmann. 8. 15 S. [Sonderabdruck aus dessen Ztschr. f. wiss. Zool. Bd. 14. Hft. 2. p. 174.]
\end{enumerate}
\end{document}
